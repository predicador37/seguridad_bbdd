%%%%%%%%%%%%%%%%%%%%%%%%%%%%%%%%%%%%%%%%%
% Beamer Presentation
% LaTeX Template
% Version 1.0 (10/11/12)
%
% This template has been downloaded from:
% http://www.LaTeXTemplates.com
%
% License:
% CC BY-NC-SA 3.0 (http://creativecommons.org/licenses/by-nc-sa/3.0/)
%
%%%%%%%%%%%%%%%%%%%%%%%%%%%%%%%%%%%%%%%%%

%----------------------------------------------------------------------------------------
%	PACKAGES AND THEMES
%----------------------------------------------------------------------------------------

\documentclass[10pt, spanish]{beamer}


\mode<presentation> {
	
	% The Beamer class comes with a number of default slide themes
	% which change the colors and layouts of slides. Below this is a list
	% of all the themes, uncomment each in turn to see what they look like.
	
	%\usetheme{default}
	%\usetheme{AnnArbor}
	%\usetheme{Antibes}
	%\usetheme{Bergen}
	%\usetheme{Berkeley}
	%\usetheme{Berlin}
	%\usetheme{Boadilla}
	%\usetheme{CambridgeUS}
	%\usetheme{Copenhagen}
	%\usetheme{Darmstadt}
	%\usetheme{Dresden}
	%\usetheme{Frankfurt}
	%\usetheme{Goettingen}
	%\usetheme{Hannover}
	%\usetheme{Ilmenau}
	%\usetheme{JuanLesPins}
	%\usetheme{Luebeck}
	%\usetheme{Madrid}
	%\usetheme{Malmoe}
	%\usetheme{Marburg}
	%\usetheme{Montpellier}
	%\usetheme{PaloAlto}
	%\usetheme{Pittsburgh}
	\usetheme{Rochester}
	%\usetheme{Singapore}
	%\usetheme{Szeged}
	%\usetheme{Warsaw}
	
	% As well as themes, the Beamer class has a number of color themes
	% for any slide theme. Uncomment each of these in turn to see how it
	% changes the colors of your current slide theme.
	
	%\usecolortheme{albatross}
	%\usecolortheme{beaver}
	%\usecolortheme{beetle}
	%\usecolortheme{crane}
	%\usecolortheme{dolphin}
	%\usecolortheme{dove}
	%\usecolortheme{fly}
	%\usecolortheme{lily}
	%\usecolortheme{orchid}
	%\usecolortheme{rose}
	%\usecolortheme{seagull}
	%\usecolortheme{seahorse}
	%\usecolortheme{whale}
	%\usecolortheme{wolverine}
	
	%\setbeamertemplate{footline} % To remove the footer line in all slides uncomment this line
	%\setbeamertemplate{footline}[page number] % To replace the footer line in all slides with a simple slide count uncomment this line
	
	%\setbeamertemplate{navigation symbols}{} % To remove the navigation symbols from the bottom of all slides uncomment this line
}

\usepackage{graphicx} % Allows including images
\usepackage{booktabs} % Allows the use of \toprule, \midrule and \bottomrule in tables
\usepackage[latin9]{inputenc}
\usepackage[spanish]{babel}
\usepackage{svg}
\usepackage{caption}
\usepackage{minted}
\captionsetup{font=scriptsize,labelfont=scriptsize}
\setbeamerfont{footnote}{size=\tiny}
\hypersetup{colorlinks=true,linkcolor=blue, linktocpage}
\graphicspath{ {./img/} }
\svgpath{ {./img/} }

%----------------------------------------------------------------------------------------
%	TITLE PAGE
%----------------------------------------------------------------------------------------

\title[Short title]{Certificaci�n y controles de seguridad en Bases de Datos: Sistemas de gesti�n de bases de datos} % The short title appears at the bottom of every slide, the full title is only on the title page

\author{Miguel Exp�sito Mart�n} % Your name
\institute[UNICAN] % Your institution as it will appear on the bottom of every slide, may be shorthand to save space
{
	Universidad de Cantabria \\ % Your institution for the title page
	\medskip
	\textit{miguel.exposito@unican.es} % Your email address
}
\date{26/11/2018} % Date, can be changed to a custom date

\addtobeamertemplate{navigation symbols}{}{%
	\usebeamerfont{footline}%
	\usebeamercolor[fg]{footline}%
	\hspace{1em}%
	\insertframenumber/\inserttotalframenumber
}

\begin{document}
	
	\begin{frame}
		\titlepage % Print the title page as the first slide
	\end{frame}
	
	\begin{frame}
		\frametitle{Visi�n general} % Table of contents slide, comment this block out to remove it
		\tableofcontents % Throughout your presentation, if you choose to use \section{} and \subsection{} commands, these will automatically be printed on this slide as an overview of your presentation
	\end{frame}

%----------------------------------------------------------------------------------------
%	PRESENTATION SLIDES
%----------------------------------------------------------------------------------------

%------------------------------------------------
\section{Definici�n} % Sections can be created in order to organize your presentation into discrete blocks, all sections and subsections are automatically printed in the table of contents as an overview of the talk
%------------------------------------------------


\begin{frame}
\frametitle{Definici�n}

Structured Query Language, o SQL, es el lenguaje utilizado tanto para la definici�n como para la manipulaci�n de datos en los SGBD relacionales. Se trata de uno de los lenguajes de bases de datos m�s populares y en uso en la industria. Entre sus caracter�sticas se pueden destacar:
\begin{itemize}
	\item Presente desde 1986, la versi�n m�s reciente del est�ndar ISO es SQL:2016.
	\item Cada proveedor implementa su dialecto particular, si bien el n�cleo del est�ndar es soportado por todos.
	\item Es un lenguaje declarativo y orientado a conjuntos.
	\item Puede utilizarse directamente desde una herramienta o consola o bien a trav�s de otro lenguaje de programaci�n.
	
\end{itemize}

\end{frame}

\section{SQL b�sico} % Sections can be created in order to organize your presentation into discrete blocks, all sections and subsections are automatically printed in the table of contents as an overview of the talk

\begin{frame}
	\frametitle{Comandos administrativos}
	\begin{itemize}
		
		\item \mintinline{sql}{USE [database name]}: establece la base de datos actual.
		\item \mintinline{sql}{SHOW DATABASES}: muestra las bases de datos existentes.	
		\item \mintinline{sql}{SHOW TABLES}: muestra todas las tablas no temporales.
		\item \mintinline{sql}{SHOW COLUMNS FROM [table name]}: proporciona informaci�n sobre las columnas de una determinada tabla.
		\item \mintinline{sql}{SHOW INDEX FROM TABLENAME [table name]}: proporciona informaci�n sobre los �ndices de una determinada tabla.
		\item \mintinline{sql}{SHOW TABLE STATUS LIKE [table name]\G}: proporciona m�s informaci�n sobre tablas no temporales utilizando el patr�n despu�s del LIKE.
		
	\end{itemize}
\end{frame}



\begin{frame}[fragile]
	\frametitle{DDL}
\setbeamerfont{block body}{size=\small}

\begin{block}{Creaci�n de bases de datos y tablas}
	\begin{minted}{text}
CREATE [OR REPLACE] {DATABASE | SCHEMA} [IF NOT EXISTS] db_name
DROP {DATABASE | SCHEMA} [IF EXISTS] db_name}
CREATE [OR REPLACE] [TEMPORARY] TABLE [IF NOT EXISTS] tbl_name
	\end{minted}
\end{block}


\begin{exampleblock}{Ejemplos}
	\begin{verbatim}
	CREATE DATABASE db1;
	DROP DATABASE db1;
	create table if not exists test (
	id bigint auto_increment primary key,
	name varchar(128) charset utf8,
	key name (name(32))
	) engine=InnoDB default charset latin1;
	\end{verbatim}
\end{exampleblock}


\end{frame}

\begin{frame}[fragile]
	\frametitle{DDL}
	\setbeamerfont{block body}{size=\small}
	
	\begin{block}{Modificaci�n y borrado de tablas}
		\begin{minted}{text}
ALTER [ONLINE] [IGNORE] TABLE tbl_name
[WAIT n | NOWAIT]
alter_specification [, alter_specification] ...

DROP [TEMPORARY] TABLE [IF EXISTS] [/*COMMENT TO SAVE*/]
tbl_name [, tbl_name] ...
[WAIT n|NOWAIT]
[RESTRICT | CASCADE]
		\end{minted}
	\end{block}
	
	
	\begin{exampleblock}{Ejemplos}
		\begin{verbatim}
		ALTER TABLE t1 ADD x INT;
		ALTER TABLE t1 DROP x;
		DROP TABLE Employees, Customers;
		\end{verbatim}
	\end{exampleblock}
	
\end{frame}

\begin{frame}[fragile]
	\frametitle{DML}
	\setbeamerfont{block body}{size=\small}
	
	\begin{block}{Consulta de tablas}
		\begin{minted}{text}
SELECT
[ALL | DISTINCT | DISTINCTROW]
[HIGH_PRIORITY]
[STRAIGHT_JOIN]
[SQL_SMALL_RESULT] [SQL_BIG_RESULT] [SQL_BUFFER_RESULT]
[SQL_CACHE | SQL_NO_CACHE] [SQL_CALC_FOUND_ROWS]
select_expr [, select_expr ...]
[ FROM table_references
[WHERE where_condition]
[GROUP BY {col_name | expr | position} [ASC | DESC], ... ]
[HAVING where_condition]
[ORDER BY {col_name | expr | position} [ASC | DESC], ...]
[LIMIT {[offset,] row_count | row_count OFFSET offset}]
[PROCEDURE procedure_name(argument_list)]
[INTO OUTFILE 'file_name' [CHARACTER SET charset_name]
		\end{minted}
	\end{block}

\end{frame}

\begin{frame}[fragile]
	\frametitle{DML}
	\setbeamerfont{block body}{size=\small}

	\begin{block}{Inserci�n de datos en tablas}
		\begin{verbatim}
		INSERT [LOW_PRIORITY | DELAYED | HIGH_PRIORITY] [IGNORE]
		[INTO] tbl_name [PARTITION (partition_list)] [(col,...)]
		{VALUES | VALUE} ({expr | DEFAULT},...),(...),...
		[ ON DUPLICATE KEY UPDATE
		col=expr [, col=expr] ... ]
		\end{verbatim}
	\end{block}
	\begin{block}{Actualizaci�n de datos en tablas}
	\begin{verbatim}
	INSERT [LOW_PRIORITY | DELAYED | HIGH_PRIORITY] [IGNORE]
	[INTO] tbl_name [PARTITION (partition_list)] [(col,...)]
	{VALUES | VALUE} ({expr | DEFAULT},...),(...),...
	[ ON DUPLICATE KEY UPDATE
	col=expr
	[, col=expr] ... ]
	\end{verbatim}
\end{block}
	
\end{frame}

\begin{frame}[fragile]
	\frametitle{DML}
	\setbeamerfont{block body}{size=\small}
	
	\begin{block}{Borrado de datos en tablas}
		\begin{verbatim}
		DELETE [LOW_PRIORITY] [QUICK] [IGNORE] 
		FROM tbl_name [PARTITION (partition_list)]
		[WHERE where_condition]
		[ORDER BY ...]
		[LIMIT row_count]
		[RETURNING select_expr 
		[, select_expr ...]]
		\end{verbatim}
	\end{block}
	\begin{block}{Vaciado de tablas}
		\begin{verbatim}
	TRUNCATE [TABLE] tbl_name
	[WAIT n | NOWAIT]
		\end{verbatim}
	\end{block}

\end{frame}

\begin{frame}[fragile]
	\frametitle{Ejercicio 3.1}
	Cree una tabla provincia con los siguientes campos:
	\begin{itemize}
		\item id
		\item cautonoma\textunderscore id
		\item literal
	\end{itemize}
Aseg�rese de que la combinaci�n de columnas id y cautonoma\textunderscore id es �nica. Utilice las referencias proporcionadas para insertar valores en la tabla: \href{https://www.ine.es/daco/daco42/codmun/cod_ccaa.htm}{CCAA}, \href{https://www.ine.es/daco/daco42/codmun/cod_provincia.htm}{provincias}.
	
\end{frame}


%------------------------------------------------
\begin{frame}
	\frametitle{Resumen}
		
\end{frame}
%------------------------------------------------

\begin{frame}
\frametitle{Referencias}
\footnotesize{
\begin{thebibliography}{99} % Beamer does not support BibTeX so references must be inserted manually as below
\bibitem[Silberschatz, 2010]{p1} Abraham Silberschatz et al. (2010)
\newblock Database System Concepts, 6th edition

\bibitem[Lemahieu, 2018]{p2} Wilfried Lemahieu et al. (2018)
\newblock Principles of Database Management
\end{thebibliography}
}
\end{frame}

%------------------------------------------------


%----------------------------------------------------------------------------------------

\end{document}